\chapter{Introduction}
Cybersecurity has proven to be a key player in the digital era and, thus, an important matter to any company or government that needs to use or provide online services. The investments in cybersecurity have been arising along the years to which is expected to exceed \$124 billion USD in 2019 in order to contract cloud-based security services \cite{GrowthForecast}. However, there is no generic answer to the best way to defend against cyberattacks, since the protections offered can differ from several aspects (e.g., price, technology, performance, and supported cyberattacks). Thus, the process of decide for one cybersecurity solution or configurations is a hard task, mainly when there are no cybersecurity experts in a company.

Based on that, recommender systems with simplified interfaces can help end-users to understand the most suitable cybersecurity solution based on his real demands. However, there is a lack of this kind of system that might not be able only to identify the best solution for a network attack but also guide the end-user for possible configurations and tasks to be done in order to prevent and mitigate a cyberattack.

In order to take advantage of this system, a broad scope of possible solutions can be provided based on the existing state-of-the-art technologies such as Software-defined Networking (SDN), Network Functions Virtualization (NFV), and machine learning. Such technologies can be, for example, a base for tools that could be an issue to protect against different cyberattacks. Therefore, information from the underlying infrastructure, cyberattacks characteristics, and end-users profiles can be used as input to help in a way that eases the task of planning and protection of its business. For that, different mechanisms can be explored to create a human-computer interface where information can be obtained directly from the end-user.

In such a direction, several studies have found ways to interact in a more natural way using chatbots \cite{networkIntents} to map text to intents that might suggest the solution for a common or already experienced network problem. This is a guideline proposed as an accepted enhancer of the interaction between end-users and recommender systems to benefit different cybersecurity aspects. DialogFlow Chatbot, for example, can be used to map the needs of end-users using Natural Language Processing (NLP) to generate a structure that is capable of providing enough abstraction to be used by cybersecurity recommender systems like that introduced in \cite{MENTOR}.

\section{Motivation}
At the current state-of-the-art, there is a tendency to grow in the cloud-based services. As discussed in \cite{growth2}, the expected annual growth of the cloud computing services is about 27\% annually starting from 2018. Thus the amount of the network operators needed would be defied from the market study tendency, which shows a growth of only 24\% in computer science careers across the US \cite{computerScienceStudents}. This will undoubtedly culminate in an increasing tendency for network operators and sysadmins, which are educated to handle security issues such as cyberattacks, performance issues, or business availability. At the same time, there is a growing tendency of cyberattacks as described by [REF], thus meaning that measures to guarantee security aspects must be maintained while taking into account the limited amount of people capable of maintaining the security of a system. Thus, there are opportunities to design solutions to help in the different aspects involved in the cybersecurity decision process.

Although such kind of solutions is still scarce, there are available tools that can be used to ease the process of eliciting the requirements from a system engineer and parse them into more natural constraints, such as Lex from Amazon or DialogFlow from Google. Such solutions use NLP to learn, identify, and process inputs from users to map into a rather simple set of actions. Thus, information can be mapped, for example, in order to ease the task of deciding the best way to defend against a cyberattack. In such a direction, we argue that there are opportunities to explore NLP as chatbots to provide simplified interfaces where operators can interact to define its requirements as intents, in a simplified way, to receive technical recommendations of how to prevent or mitigate cyberattacks, such as applying configurations or deploying protection services.


\section{Description of Work}
This thesis consists of the design and implementation of a chatbot capable of interacting with
end-users using natural language and translates the end-users' intents to a data structure to be
used as input for a protection recommender system. To reach this goal, Thus, as initial input, all information related to end-user's infrastructure, budget, and cyberattack reports is considered. As output, the chatbot can provide insights regarding suitable solutions that satisfy the user's demands.

In order to achieve the thesis goals, technologies such as DialogFlow, in conjunction with a comprehensive man-page database [REF], is used. Furthermore, this work provides a web-based interface for interaction between end-users and recommendation systems like \cite{MENTOR}. The goal is to present a trained chatbot that is capable of suggesting a useful schema for a recommendation system based on entity analysis from the input of the end-user. It is important to mention that the chatbot proposed in this thesis will enable a client front-end to allow user interaction to ease the issue of defending and identifying cyberattacks.


\section{Thesis Outline}
The thesis will be described as an introductory step to background work and leading to a reference to related works, in order to understand how NLP works and how is this relevant for DialogFlow and Entity Analysis. Furthermore, an introduction to the existing tools and related work that reasambles this in some way is presented. The aim for this is to create enough background to understand the presented solution which will hold all the values recollected as following the sections of [TO BE FILLED WITH THE WORK SECTIONS].

% Example of structure for the outline:
% This thesis is organized as follow. Chapter 1 introduces and motivates the work. Next, Chapter 2 provides details of relevant background and review of the state-of-the-art. In Chapter 3, ...