\chapter{Future Work and Conclusion}

\section{Future Work}
As future improvements to our chatbot we can take into account that most models are meant to deliver a particular function, in our case we aim to extract information and provide a representation of this information in an abstract way. In our current status there are many areas that can be improved to achieve a better model and to improve the confidence interval within the context. For example, Currently we use the confidence interval to retrieve the current information on the loss function, but we use negative phrases for this even though the model was trained without negative examples. We should focus in expanding our usage to include negative training phrases to increase the accuracy of the model. This is directly proportional to the amount of accuracy we want to reach, the more information we can provide to the model the better it will get.

Another interesting aspect in which we could improve the functionality of the bot is the way it interacts with the Client, at the moment we are only using the controller to pass data and do flow control. It could also handle elements that are integrated within the DialogFlow implementation itself such as providing localization, speech to text features and parsing of other information such as images and focalized data. This is all integrated within the DialogFlow API and is simple to setup and use, we could potentially increase the usability of the chatbot by providing google searches or even images directly for identification and solutions.

Also, the amount of information that we are using for the NER is almost naggable compared to the total information that lies within the datastore of DialogFlow. For this we could use the already existent parser written in python to extract elements such as the Synopsis and create another category of Entities based on this. In this way we could have entity identification for a broader set of tooling. Also, having a larger dataset will reduce the probability of missing the identification of an entity and since the dataset that we currently have is related to computer science or engineering the information is relevant and labeled.

Finally, we could improve the measurements in the evaluation section by lowering the confidence interval to 35\% or even lower, this is because at the moment we evaluated with a model above 40\% and within the identified data many intents were correctly identified meaning that we had false negatives reducing the real metrics of the model. We would need to test and trial a right number that would reflect the real accuracy of the model but this would also mean a better way to create intent phrases in our model. All these improvements can benefit the most by using an opensource model that would allow us to tweak each element at it's core such as RASA NLU which can be beneficial in our case since it features all the current capabilities of DialogFlow but the pretrained model, which in case we manage to gather enough information would not be needed anymore and thus we could modify the source code and the intent classifier to use cutting edge algorithms that would most likely yield better results.

\section{Conclusion}

In our work we decided to take available information from different sources such as man pages and white papers to generate a database of content that could be fed into a Machine Learning algorithm that would allow us to parse queries from a user and predict if they match a certain intent predefined. In order to achieve this, we used a classifier that featured a loss function that would relate to our product. The results show that this is a feasible task but requires a lot of information in order to obtain accurate results. Also, we defined a structure that could be relevant for recommender systems such as MENTOR to perform the data extraction and have a generic abstraction of information from a certain user.

We can have several areas into which this can be expanded and are subject to evaluation. Our results point out that the best method to achieve this once enough information is available is to train a supervised classifier and enhance it with trending techniques such as reinforcement learning. This is a relevant area and has quite potential, the client is capable of handling all requests due the architecture and furthermore we can integrate new features into the client in a relatively easy manner.

To be Continued/ Finished